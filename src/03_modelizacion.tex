% Options for packages loaded elsewhere
\PassOptionsToPackage{unicode}{hyperref}
\PassOptionsToPackage{hyphens}{url}
%
\documentclass[
]{article}
\usepackage{lmodern}
\usepackage{amssymb,amsmath}
\usepackage{ifxetex,ifluatex}
\ifnum 0\ifxetex 1\fi\ifluatex 1\fi=0 % if pdftex
  \usepackage[T1]{fontenc}
  \usepackage[utf8]{inputenc}
  \usepackage{textcomp} % provide euro and other symbols
\else % if luatex or xetex
  \usepackage{unicode-math}
  \defaultfontfeatures{Scale=MatchLowercase}
  \defaultfontfeatures[\rmfamily]{Ligatures=TeX,Scale=1}
\fi
% Use upquote if available, for straight quotes in verbatim environments
\IfFileExists{upquote.sty}{\usepackage{upquote}}{}
\IfFileExists{microtype.sty}{% use microtype if available
  \usepackage[]{microtype}
  \UseMicrotypeSet[protrusion]{basicmath} % disable protrusion for tt fonts
}{}
\makeatletter
\@ifundefined{KOMAClassName}{% if non-KOMA class
  \IfFileExists{parskip.sty}{%
    \usepackage{parskip}
  }{% else
    \setlength{\parindent}{0pt}
    \setlength{\parskip}{6pt plus 2pt minus 1pt}}
}{% if KOMA class
  \KOMAoptions{parskip=half}}
\makeatother
\usepackage{xcolor}
\IfFileExists{xurl.sty}{\usepackage{xurl}}{} % add URL line breaks if available
\IfFileExists{bookmark.sty}{\usepackage{bookmark}}{\usepackage{hyperref}}
\hypersetup{
  pdftitle={Modelización de calidad del aire},
  hidelinks,
  pdfcreator={LaTeX via pandoc}}
\urlstyle{same} % disable monospaced font for URLs
\usepackage[margin=1in]{geometry}
\usepackage{graphicx,grffile}
\makeatletter
\def\maxwidth{\ifdim\Gin@nat@width>\linewidth\linewidth\else\Gin@nat@width\fi}
\def\maxheight{\ifdim\Gin@nat@height>\textheight\textheight\else\Gin@nat@height\fi}
\makeatother
% Scale images if necessary, so that they will not overflow the page
% margins by default, and it is still possible to overwrite the defaults
% using explicit options in \includegraphics[width, height, ...]{}
\setkeys{Gin}{width=\maxwidth,height=\maxheight,keepaspectratio}
% Set default figure placement to htbp
\makeatletter
\def\fps@figure{htbp}
\makeatother
\setlength{\emergencystretch}{3em} % prevent overfull lines
\providecommand{\tightlist}{%
  \setlength{\itemsep}{0pt}\setlength{\parskip}{0pt}}
\setcounter{secnumdepth}{-\maxdimen} % remove section numbering

\title{Modelización de calidad del aire}
\usepackage{etoolbox}
\makeatletter
\providecommand{\subtitle}[1]{% add subtitle to \maketitle
  \apptocmd{\@title}{\par {\large #1 \par}}{}{}
}
\makeatother
\subtitle{MSMK-Machine learning}
\author{}
\date{\vspace{-2.5em}Octubre 2020}

\begin{document}
\maketitle

En esta práctica vamos a utilizar los datos de \textbf{calidad del aire}
para aplicar los modelos de minería de datos que conocemos.

\begin{quote}
\textbf{El objetivo es predecir el nivel de \texttt{pm25} para el día
siguiente}. Esto permitiría, por ejemplo, al ayuntamiento activar los
protocolos anticontaminación con la suficiente antelación.
\end{quote}

\hypertarget{paso-1}{%
\section{Paso 1}\label{paso-1}}

\begin{enumerate}
\def\labelenumi{\arabic{enumi}.}
\tightlist
\item
  Importa los datos de \texttt{train.RDS} y \texttt{test.RDS}.
\item
  La primera observación de \texttt{train} no tendrá sentido ya que
  \textbf{todas las variables \texttt{\_lag} aparecerán como
  \texttt{NA}}. Elimina esta primera observación de \texttt{train}.
\item
  Elimina la columna fecha de \texttt{train} y \texttt{test}.
\end{enumerate}

\hypertarget{paso-2}{%
\section{Paso 2}\label{paso-2}}

\begin{enumerate}
\def\labelenumi{\arabic{enumi}.}
\tightlist
\item
  Representa en un \textbf{diagrama de dispersión} (recuerda que en
  ggplot tienes que usar \texttt{geom\_point}) la relación entre ambas
  variables. Como es habitual, en el eje x representa la variable
  predictora \texttt{pm25\_lag} y el eje y, la variable objetivo
  \texttt{pm25}.
\end{enumerate}

En muchos proyectos, antes de abordar la modelización con modelos
complejos, se comienza realizando una modelización muy simple que se
suele denominar \emph{baseline} y que nos permite compararla con otros
modelos más complejos que podamos utilizar así como hacernos la pregunta
de si vale la pena ese esfuerzo.

\begin{enumerate}
\def\labelenumi{\arabic{enumi}.}
\setcounter{enumi}{1}
\tightlist
\item
  Genera un modelo denominado \texttt{mod\_baseline} que sea una
  regresión lineal (utiliza la función \texttt{lm}) que prediga
  \texttt{pm25} utilizando solamente \texttt{pm25\_lag}. Genera la
  predicción de test en un objeto llamado \texttt{pred\_baseline},
  calcula su RMSE y guárdalo en una variable llamada
  \texttt{rmse\_baseline} (puedes utilizar \texttt{source} para importar
  la función \texttt{rmse} que hemos creado en clase).
\end{enumerate}

\hypertarget{paso-3}{%
\section{Paso 3}\label{paso-3}}

\begin{enumerate}
\def\labelenumi{\arabic{enumi}.}
\item
  Aplica un modelo de \texttt{bagging} sobre \texttt{train}. Prueba
  valores 50, 100, 150, 200 para el número de iteraciones. Calcula el
  rmse en test para cada modelo de bagging y quédate con el mejor
  modelo. Llama a ese modelo \texttt{mod\_bagging}, a su predicción
  \texttt{pred\_bagging} y a su RMSE \texttt{rmse\_bagging}.
\item
  Aplica un modelo de random forest denominado \texttt{mod\_rf}. Genera
  valores de \texttt{ntree} de 100 a 500 de 100 en 100. Genera valores
  de \texttt{mtry} de 2 al número de columnas de train de 3 en 3.
  Quédate con el mejor modelo y, de forma similar al apartado anterior,
  utiliza la nomenclatura \texttt{mod\_rf}, \texttt{pred\_rf} y
  \texttt{rmse\_rf}.
\item
  Aplica un modelo de boosting denominado \texttt{mod\_boost}.
  Utilizando los parámetros \texttt{eta}, \texttt{max\_depth},
  \texttt{subsample}, \texttt{colsample\_bytree}, genera un mínimo de 50
  combinaciones (en total) y quédate con la mejor. De nuevo, utiliza la
  nomenclatura \texttt{pred\_boost} y \texttt{rmse\_boost}. Utiliza un
  \texttt{early\ stopping} de 50 y \texttt{nrounds} de 1000
\end{enumerate}

\hypertarget{paso-4}{%
\section{Paso 4}\label{paso-4}}

Compara los \(RMSE\) de los 4 modelos que hemos generado
(\emph{baseline}, bagging, random forest y boosting). ¿Cuál predice
mejor?

En este caso, al tratarse de unos datos temporales, podemos representar
fácilmente la información. Genera un gráfico \emph{similar} al siguiente
en el que se compare el valor real en test con el predicho por
\textbf{cada modelo}, es decir, debes hacer \textbf{un gráfico para
baseline, bagging, random forest y boosting}.

\includegraphics{03_modelizacion_files/figure-latex/unnamed-chunk-3-1.pdf}

\begin{quote}
\textbf{Nota:} en el gráfico anterior se ha modificado la apariencia
básica de ggplot2.
\end{quote}

\hypertarget{paso-5}{%
\section{Paso 5}\label{paso-5}}

Vamos a enriquecer el conjunto de datos para ver si podemos mejorar la
predicción.

\begin{enumerate}
\def\labelenumi{\arabic{enumi}.}
\tightlist
\item
  Descarga el conjunto de datos \texttt{dias\_laborables.RDS}. Importa
  este conjunto de datos (recuerda que tiene un formato \texttt{.RDS}).
  Este \texttt{data.frame} contiene una variable \texttt{fecha} y otra
  variable \texttt{laborable} que toma valor \(1\) si es un día
  laborable o \(0\) si es un festivo, sábado o domingo.
\item
  Añade esta información a \texttt{train} y \texttt{test} utilizando la
  función \texttt{left\_join()} de forma adecuada.
\item
  Reentrena los modelos \texttt{baseline}, \texttt{bagging},
  \texttt{random\ forest} y \texttt{boosting} con esta nueva información
  (para el modelo baseline utiliza
  \texttt{pm25\ \textasciitilde{}\ pm25\_lag\ +\ laborable}). Llama a
  cada modelo igual que hicimos antes pero terminado en 2 (por ejemplo,
  \texttt{mod\_boost2}, \texttt{pred\_boost2}, \texttt{rmse\_boost2}).
\item
  Calcula y representa gráficamente la importancia de las variables del
  modelo de boosting. ¿Es relevante que un día sea laborable para el
  modelo?
\item
  ¿Se mejoran los valores de \(RMSE\) conseguidos anteriormente? Para
  contestar a la pregunta genera un data frame con una columna
  \texttt{modelo} y otra \texttt{rmse} donde se recojan los valores de
  forma similar a la tabla siguiente (los valores de la tabla del
  ejemplo no tienen por qué ser reales). Guarda este data frame como un
  archivo \texttt{.RDS} utilizando la función \texttt{writeRDS()}.
\end{enumerate}

\begin{verbatim}
## # A tibble: 8 x 2
##   modelo          rmse
##   <chr>          <dbl>
## 1 baseline        3.21
## 2 baseline2       3.20
## 3 bagging         2.92
## 4 bagging2        2.88
## 5 random forest   2.94
## 6 random forest2  2.91
## 7 boosting        2.91
## 8 boosting2       2.89
\end{verbatim}

\end{document}
