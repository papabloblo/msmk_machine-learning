\PassOptionsToPackage{unicode=true}{hyperref} % options for packages loaded elsewhere
\PassOptionsToPackage{hyphens}{url}
%
\documentclass[]{article}
\usepackage{lmodern}
\usepackage{amssymb,amsmath}
\usepackage{ifxetex,ifluatex}
\usepackage{fixltx2e} % provides \textsubscript
\ifnum 0\ifxetex 1\fi\ifluatex 1\fi=0 % if pdftex
  \usepackage[T1]{fontenc}
  \usepackage[utf8]{inputenc}
  \usepackage{textcomp} % provides euro and other symbols
\else % if luatex or xelatex
  \usepackage{unicode-math}
  \defaultfontfeatures{Ligatures=TeX,Scale=MatchLowercase}
\fi
% use upquote if available, for straight quotes in verbatim environments
\IfFileExists{upquote.sty}{\usepackage{upquote}}{}
% use microtype if available
\IfFileExists{microtype.sty}{%
\usepackage[]{microtype}
\UseMicrotypeSet[protrusion]{basicmath} % disable protrusion for tt fonts
}{}
\IfFileExists{parskip.sty}{%
\usepackage{parskip}
}{% else
\setlength{\parindent}{0pt}
\setlength{\parskip}{6pt plus 2pt minus 1pt}
}
\usepackage{hyperref}
\hypersetup{
            pdftitle={2. Preparación para modelización},
            pdfauthor={Minería de datos II},
            pdfborder={0 0 0},
            breaklinks=true}
\urlstyle{same}  % don't use monospace font for urls
\usepackage[margin=1in]{geometry}
\usepackage{color}
\usepackage{fancyvrb}
\newcommand{\VerbBar}{|}
\newcommand{\VERB}{\Verb[commandchars=\\\{\}]}
\DefineVerbatimEnvironment{Highlighting}{Verbatim}{commandchars=\\\{\}}
% Add ',fontsize=\small' for more characters per line
\usepackage{framed}
\definecolor{shadecolor}{RGB}{248,248,248}
\newenvironment{Shaded}{\begin{snugshade}}{\end{snugshade}}
\newcommand{\AlertTok}[1]{\textcolor[rgb]{0.94,0.16,0.16}{#1}}
\newcommand{\AnnotationTok}[1]{\textcolor[rgb]{0.56,0.35,0.01}{\textbf{\textit{#1}}}}
\newcommand{\AttributeTok}[1]{\textcolor[rgb]{0.77,0.63,0.00}{#1}}
\newcommand{\BaseNTok}[1]{\textcolor[rgb]{0.00,0.00,0.81}{#1}}
\newcommand{\BuiltInTok}[1]{#1}
\newcommand{\CharTok}[1]{\textcolor[rgb]{0.31,0.60,0.02}{#1}}
\newcommand{\CommentTok}[1]{\textcolor[rgb]{0.56,0.35,0.01}{\textit{#1}}}
\newcommand{\CommentVarTok}[1]{\textcolor[rgb]{0.56,0.35,0.01}{\textbf{\textit{#1}}}}
\newcommand{\ConstantTok}[1]{\textcolor[rgb]{0.00,0.00,0.00}{#1}}
\newcommand{\ControlFlowTok}[1]{\textcolor[rgb]{0.13,0.29,0.53}{\textbf{#1}}}
\newcommand{\DataTypeTok}[1]{\textcolor[rgb]{0.13,0.29,0.53}{#1}}
\newcommand{\DecValTok}[1]{\textcolor[rgb]{0.00,0.00,0.81}{#1}}
\newcommand{\DocumentationTok}[1]{\textcolor[rgb]{0.56,0.35,0.01}{\textbf{\textit{#1}}}}
\newcommand{\ErrorTok}[1]{\textcolor[rgb]{0.64,0.00,0.00}{\textbf{#1}}}
\newcommand{\ExtensionTok}[1]{#1}
\newcommand{\FloatTok}[1]{\textcolor[rgb]{0.00,0.00,0.81}{#1}}
\newcommand{\FunctionTok}[1]{\textcolor[rgb]{0.00,0.00,0.00}{#1}}
\newcommand{\ImportTok}[1]{#1}
\newcommand{\InformationTok}[1]{\textcolor[rgb]{0.56,0.35,0.01}{\textbf{\textit{#1}}}}
\newcommand{\KeywordTok}[1]{\textcolor[rgb]{0.13,0.29,0.53}{\textbf{#1}}}
\newcommand{\NormalTok}[1]{#1}
\newcommand{\OperatorTok}[1]{\textcolor[rgb]{0.81,0.36,0.00}{\textbf{#1}}}
\newcommand{\OtherTok}[1]{\textcolor[rgb]{0.56,0.35,0.01}{#1}}
\newcommand{\PreprocessorTok}[1]{\textcolor[rgb]{0.56,0.35,0.01}{\textit{#1}}}
\newcommand{\RegionMarkerTok}[1]{#1}
\newcommand{\SpecialCharTok}[1]{\textcolor[rgb]{0.00,0.00,0.00}{#1}}
\newcommand{\SpecialStringTok}[1]{\textcolor[rgb]{0.31,0.60,0.02}{#1}}
\newcommand{\StringTok}[1]{\textcolor[rgb]{0.31,0.60,0.02}{#1}}
\newcommand{\VariableTok}[1]{\textcolor[rgb]{0.00,0.00,0.00}{#1}}
\newcommand{\VerbatimStringTok}[1]{\textcolor[rgb]{0.31,0.60,0.02}{#1}}
\newcommand{\WarningTok}[1]{\textcolor[rgb]{0.56,0.35,0.01}{\textbf{\textit{#1}}}}
\usepackage{longtable,booktabs}
% Fix footnotes in tables (requires footnote package)
\IfFileExists{footnote.sty}{\usepackage{footnote}\makesavenoteenv{longtable}}{}
\usepackage{graphicx,grffile}
\makeatletter
\def\maxwidth{\ifdim\Gin@nat@width>\linewidth\linewidth\else\Gin@nat@width\fi}
\def\maxheight{\ifdim\Gin@nat@height>\textheight\textheight\else\Gin@nat@height\fi}
\makeatother
% Scale images if necessary, so that they will not overflow the page
% margins by default, and it is still possible to overwrite the defaults
% using explicit options in \includegraphics[width, height, ...]{}
\setkeys{Gin}{width=\maxwidth,height=\maxheight,keepaspectratio}
\setlength{\emergencystretch}{3em}  % prevent overfull lines
\providecommand{\tightlist}{%
  \setlength{\itemsep}{0pt}\setlength{\parskip}{0pt}}
\setcounter{secnumdepth}{5}
% Redefines (sub)paragraphs to behave more like sections
\ifx\paragraph\undefined\else
\let\oldparagraph\paragraph
\renewcommand{\paragraph}[1]{\oldparagraph{#1}\mbox{}}
\fi
\ifx\subparagraph\undefined\else
\let\oldsubparagraph\subparagraph
\renewcommand{\subparagraph}[1]{\oldsubparagraph{#1}\mbox{}}
\fi

% set default figure placement to htbp
\makeatletter
\def\fps@figure{htbp}
\makeatother

\usepackage{etoolbox}
\makeatletter
\providecommand{\subtitle}[1]{% add subtitle to \maketitle
  \apptocmd{\@title}{\par {\large #1 \par}}{}{}
}
\makeatother

\title{2. Preparación para modelización}
\providecommand{\subtitle}[1]{}
\subtitle{Taller: calidad del aire}
\author{Minería de datos II}
\date{Curso 2019/2020}

\begin{document}
\maketitle

{
\setcounter{tocdepth}{2}
\tableofcontents
}
\hypertarget{entorno}{%
\section{Entorno}\label{entorno}}

Crea un \textbf{script R} en la carpeta
\texttt{taller\_calidad\_aire/src} que se llame
\texttt{02\_preparacion\_modelizacion.R}; será el script que
desarrollaremos en este documento.

Al principio del script, carga el paquete \texttt{tidyverse}:

\begin{Shaded}
\begin{Highlighting}[]
\KeywordTok{library}\NormalTok{(tidyverse)}
\end{Highlighting}
\end{Shaded}

\begin{verbatim}
## -- Attaching packages --------------------------------------------- tidyverse 1.3.0 --
\end{verbatim}

\begin{verbatim}
## v ggplot2 3.2.1     v purrr   0.3.3
## v tibble  2.1.3     v dplyr   0.8.3
## v tidyr   1.0.0     v stringr 1.4.0
## v readr   1.3.1     v forcats 0.4.0
\end{verbatim}

\begin{verbatim}
## -- Conflicts ------------------------------------------------ tidyverse_conflicts() --
## x dplyr::filter() masks stats::filter()
## x dplyr::lag()    masks stats::lag()
\end{verbatim}

\hypertarget{importaciuxf3n-de-datos}{%
\section{Importación de datos}\label{importaciuxf3n-de-datos}}

Para comenzar a trabajar, necesitamos importar los datos que obtuvimos
de la fase anterior. Como lo que guardamos fue un archivo \texttt{.RDS},
los datos ya serán un \texttt{data.frame} de R y, por tanto, no tenemos
que preocuparnos del formato de los datos como teníamos que hacerlo con
un \texttt{.csv}.

\begin{Shaded}
\begin{Highlighting}[]
\NormalTok{calidad_aire <-}\StringTok{ }\KeywordTok{readRDS}\NormalTok{(}\StringTok{"taller_calidad_aire/data/01_transformacion.RDS"}\NormalTok{)}
\end{Highlighting}
\end{Shaded}

\hypertarget{anuxe1lisis-exploratorio-de-datos}{%
\section{Análisis exploratorio de
datos}\label{anuxe1lisis-exploratorio-de-datos}}

Antes de realizar cualquier transformación sobre los datos, necesitamos
conocerlos en profundidad. En cualquier proyecto de minería de datos,
una de las primeras fases una vez que los datos se han importado
correctamente es hacer un \textbf{análisis exploratorio de datos}
(\emph{EDA}).

Un paquete que nos puede ayudar a esto es el paquete \texttt{skimr}.
Este paquete nos da un resumen de los datos.

\begin{Shaded}
\begin{Highlighting}[]
\NormalTok{skimr}\OperatorTok{::}\KeywordTok{skim}\NormalTok{(calidad_aire)}
\end{Highlighting}
\end{Shaded}

\begin{longtable}[]{@{}ll@{}}
\caption{Data summary}\tabularnewline
\toprule
\endhead
Name & calidad\_aire\tabularnewline
Number of rows & 1064\tabularnewline
Number of columns & 18\tabularnewline
\_\_\_\_\_\_\_\_\_\_\_\_\_\_\_\_\_\_\_\_\_\_\_ &\tabularnewline
Column type frequency: &\tabularnewline
Date & 1\tabularnewline
numeric & 17\tabularnewline
\_\_\_\_\_\_\_\_\_\_\_\_\_\_\_\_\_\_\_\_\_\_\_\_ &\tabularnewline
Group variables & None\tabularnewline
\bottomrule
\end{longtable}

\textbf{Variable type: Date}

\begin{longtable}[]{@{}lrrlllr@{}}
\toprule
skim\_variable & n\_missing & complete\_rate & min & max & median &
n\_unique\tabularnewline
\midrule
\endhead
fecha & 0 & 1 & 2017-01-01 & 2019-11-30 & 2018-06-16 &
1064\tabularnewline
\bottomrule
\end{longtable}

\textbf{Variable type: numeric}

\begin{longtable}[]{@{}lrrrrrrrrr@{}}
\toprule
skim\_variable & n\_missing & complete\_rate & mean & sd & p0 & p25 &
p50 & p75 & p100\tabularnewline
\midrule
\endhead
ano & 0 & 1 & 2017.97 & 0.81 & 2017.00 & 2017.00 & 2018.00 & 2019.00 &
2019.00\tabularnewline
mes & 0 & 1 & 6.37 & 3.37 & 1.00 & 3.00 & 6.00 & 9.00 &
12.00\tabularnewline
dia & 0 & 1 & 15.71 & 8.80 & 1.00 & 8.00 & 16.00 & 23.00 &
31.00\tabularnewline
so2 & 0 & 1 & 7.28 & 2.49 & 2.50 & 5.30 & 7.00 & 8.57 &
18.20\tabularnewline
co & 0 & 1 & 0.35 & 0.13 & 0.14 & 0.26 & 0.31 & 0.40 &
1.00\tabularnewline
no & 0 & 1 & 18.70 & 23.97 & 1.58 & 5.12 & 8.69 & 21.01 &
171.75\tabularnewline
no2 & 0 & 1 & 37.44 & 17.34 & 8.83 & 24.54 & 33.56 & 47.26 &
104.33\tabularnewline
pm25 & 0 & 1 & 9.93 & 5.14 & 2.40 & 6.00 & 9.20 & 12.68 &
40.33\tabularnewline
pm10 & 0 & 1 & 18.58 & 10.37 & 2.58 & 10.98 & 17.00 & 23.55 &
144.17\tabularnewline
nox & 0 & 1 & 66.11 & 52.38 & 12.08 & 32.42 & 47.34 & 79.29 &
361.83\tabularnewline
o3 & 0 & 1 & 51.24 & 22.99 & 2.57 & 33.68 & 54.82 & 70.11 &
105.36\tabularnewline
tol & 0 & 1 & 2.11 & 1.72 & 0.20 & 1.01 & 1.60 & 2.55 &
13.52\tabularnewline
ben & 0 & 1 & 0.49 & 0.33 & 0.12 & 0.27 & 0.38 & 0.60 &
2.20\tabularnewline
ebe & 0 & 1 & 0.46 & 0.51 & 0.10 & 0.18 & 0.28 & 0.50 &
3.45\tabularnewline
tch & 0 & 1 & 1.43 & 0.12 & 1.11 & 1.35 & 1.41 & 1.48 &
2.05\tabularnewline
ch4 & 0 & 1 & 1.33 & 0.10 & 0.98 & 1.27 & 1.32 & 1.37 &
1.91\tabularnewline
nmhc & 0 & 1 & 0.10 & 0.04 & 0.01 & 0.07 & 0.10 & 0.13 &
0.31\tabularnewline
\bottomrule
\end{longtable}

\begin{quote}
\textbf{Nota:} recuerda que al escribir \texttt{skimr::skim} le estamos
diciendo a R que utilice la función \texttt{skim} que pertence al
paquete \texttt{skimr}. Esto es útil cuando queremos utilizar la función
\texttt{skim} una única vez y, por lo tanto, no hace falta hacer
\texttt{library(skimr)}.
\end{quote}

Si te fijas en la columan \texttt{m\_missing} del resultado anterior,
sabemos que no tenemos ningún dato ausente en ninguna variable. Por lo
tanto, no hará falta recurrir a la imputación de valores ausentes.

En el apartado de la variable fecha de la salida anterior podemos ver
que los datos van desde el \texttt{2017-01-01} hasta
\texttt{2019-11-30}.

La variable que queremos predecir es \texttt{pm25}, así que vamos a
estudiarla en más detalle.

\hypertarget{variable-pm25}{%
\subsection{\texorpdfstring{Variable
\texttt{pm25}}{Variable pm25}}\label{variable-pm25}}

\begin{Shaded}
\begin{Highlighting}[]
\KeywordTok{ggplot}\NormalTok{(}\DataTypeTok{data =}\NormalTok{ calidad_aire,}
       \KeywordTok{aes}\NormalTok{(}\DataTypeTok{x =}\NormalTok{ fecha, }\DataTypeTok{y =}\NormalTok{ pm25)) }\OperatorTok{+}\StringTok{ }
\StringTok{  }\KeywordTok{geom_line}\NormalTok{()}
\end{Highlighting}
\end{Shaded}

\includegraphics{02_preparacion_modelizacion_files/figure-latex/unnamed-chunk-5-1.pdf}

En el gráfico anterior de puede apreciar un patrón \emph{ondulante} que
parece repetirse anualmente. Podemos verlo de forma más clara si
añadimos la capa \texttt{geom\_smooth()}

\begin{Shaded}
\begin{Highlighting}[]
\KeywordTok{ggplot}\NormalTok{(}\DataTypeTok{data =}\NormalTok{ calidad_aire,}
       \KeywordTok{aes}\NormalTok{(}\DataTypeTok{x =}\NormalTok{ fecha, }\DataTypeTok{y =}\NormalTok{ pm25)) }\OperatorTok{+}\StringTok{ }
\StringTok{  }\KeywordTok{geom_line}\NormalTok{() }\OperatorTok{+}
\StringTok{  }\KeywordTok{geom_smooth}\NormalTok{()}
\end{Highlighting}
\end{Shaded}

\begin{verbatim}
## `geom_smooth()` using method = 'gam' and formula 'y ~ s(x, bs = "cs")'
\end{verbatim}

\includegraphics{02_preparacion_modelizacion_files/figure-latex/unnamed-chunk-6-1.pdf}

La distribución de la variable podemos verla mediante un histograma:

\begin{Shaded}
\begin{Highlighting}[]
\KeywordTok{ggplot}\NormalTok{(}\DataTypeTok{data =}\NormalTok{ calidad_aire,}
       \KeywordTok{aes}\NormalTok{(}\DataTypeTok{x =}\NormalTok{ pm25)) }\OperatorTok{+}\StringTok{ }
\StringTok{  }\KeywordTok{geom_histogram}\NormalTok{()}
\end{Highlighting}
\end{Shaded}

\begin{verbatim}
## `stat_bin()` using `bins = 30`. Pick better value with `binwidth`.
\end{verbatim}

\includegraphics{02_preparacion_modelizacion_files/figure-latex/unnamed-chunk-7-1.pdf}

\hypertarget{construcciuxf3n-de-variables}{%
\section{Construcción de variables}\label{construcciuxf3n-de-variables}}

Recordemos que el objetivo es \textbf{predecir el valor de \texttt{pm25}
en el día posterior}. Eso significa que, si queremos predecir el valor
para el 9 de febrero, tenemos que suponer que \textbf{solamente
conocemos la información hasta el día 8 de febrero}. Por lo tanto, el
valor que podremos utilizar de las variables para predecir un día, debe
ser la información disponible hasta el día anterior. Por ejemplo, no
podemos utilizar el valor de \texttt{so2} en el mismo día que el valor
de \texttt{pm25} que queremos predecir. Necesitamos que el valor de cada
variable esté \emph{retrasado} en un día. Esto lo podemos hacer mediante
la función \texttt{lag}. Para entenderlo, vamos hacer primero un
ejemplo. Si tuviésemos el vector \texttt{c(1,2,3,4)}, si aplicamos la
función \texttt{lag} obtendríamos

\begin{Shaded}
\begin{Highlighting}[]
\KeywordTok{lag}\NormalTok{(}\KeywordTok{c}\NormalTok{(}\DecValTok{1}\NormalTok{,}\DecValTok{2}\NormalTok{,}\DecValTok{3}\NormalTok{,}\DecValTok{4}\NormalTok{))}
\end{Highlighting}
\end{Shaded}

\begin{verbatim}
## [1] NA  1  2  3
\end{verbatim}

Obviamente, el valor anterior del primer elemento del vector es
desconocido y por eso aparece como \texttt{NA}.

Antes de utilizar la función \texttt{lag} debemos asegurarnos de que los
datos estén ordenados de menor a mayor por la variable fecha, porque en
caso contrario no tendría sentido lo que estaríamos haciendo:

\begin{Shaded}
\begin{Highlighting}[]
\NormalTok{calidad_aire <-}\StringTok{ }\KeywordTok{arrange}\NormalTok{(calidad_aire, fecha)}
\end{Highlighting}
\end{Shaded}

Y ahora generamos una nueva variable \texttt{\_lag} por cada variable
que tengamos que retrasar. Por ejemplo

\begin{Shaded}
\begin{Highlighting}[]
\NormalTok{calidad_aire}\OperatorTok{$}\NormalTok{so2_lag <-}\StringTok{ }\KeywordTok{lag}\NormalTok{(calidad_aire}\OperatorTok{$}\NormalTok{so2)}
\end{Highlighting}
\end{Shaded}

Este procedimiento habría que repetirlo demasiadas veces y sería
demasiado pesado para hacerlo de forma manual. Para \emph{automatizarlo}
podemos utilizar la función \texttt{mutate\_at}:

\begin{Shaded}
\begin{Highlighting}[]
\NormalTok{calidad_aire <-}\StringTok{ }\NormalTok{calidad_aire }\OperatorTok\StringTok{ }
\StringTok{  }\KeywordTok{mutate_at}\NormalTok{(}\KeywordTok{vars}\NormalTok{(so2}\OperatorTok{:}\NormalTok{nmhc), }\KeywordTok{list}\NormalTok{(}\DataTypeTok{lag =}\NormalTok{ lag))}
\end{Highlighting}
\end{Shaded}

\textbf{Nota}: lo que acabamos de hacer se puede traducir como: aplica
la función \texttt{lag} a aquellas variables que están entre
\texttt{so2} y \texttt{nmhc}. Al utilizar \texttt{list(lag\ =\ lag)},
cada variable que se crea termina en \texttt{\_lag}.

Para finalizar la creación de estas variables, debemos eliminar todas
las variables que no son lag y quedarnos solamente con \texttt{pm25} que
es la variable que queremos predecir.

\begin{Shaded}
\begin{Highlighting}[]
\NormalTok{calidad_aire <-}\StringTok{ }\NormalTok{calidad_aire }\OperatorTok\StringTok{ }
\StringTok{  }\KeywordTok{select}\NormalTok{(fecha, ano}\OperatorTok{:}\NormalTok{dia, so2_lag}\OperatorTok{:}\NormalTok{nmhc_lag, pm25)}
\end{Highlighting}
\end{Shaded}

Es interesante conocer la correlación de la variable objetivo con
respecto a las predictoras:

\begin{Shaded}
\begin{Highlighting}[]
\KeywordTok{cor}\NormalTok{(calidad_aire}\OperatorTok{$}\NormalTok{pm25, }
    \KeywordTok{select}\NormalTok{(calidad_aire, }\KeywordTok{ends_with}\NormalTok{(}\StringTok{"lag"}\NormalTok{)), }
    \DataTypeTok{use =} \StringTok{"complete.obs"}
\NormalTok{    )}
\end{Highlighting}
\end{Shaded}

\begin{verbatim}
##        so2_lag    co_lag    no_lag   no2_lag  pm25_lag  pm10_lag   nox_lag
## [1,] 0.4284642 0.5468669 0.4825432 0.5294305 0.6936286 0.5976506 0.5138496
##          o3_lag   tol_lag   ben_lag ebe_lag   tch_lag   ch4_lag  nmhc_lag
## [1,] -0.2627425 0.4335614 0.4198621  0.1981 0.4062614 0.3397541 0.3181111
\end{verbatim}

\begin{quote}
\textbf{Nota}: en la correlación utilizamos
\texttt{use\ =\ "complete.obs"} para que no tenga en cuenta los
\texttt{NA} en el cálculo.
\end{quote}

Puedes ver que la mayor correlación de \texttt{pm25} se da con
\texttt{pm25\_lag}.

\hypertarget{train-y-test}{%
\section{Train y test}\label{train-y-test}}

Por último, como hacemos habitualmente, vamos a dividir el conjunto de
datos en \texttt{train} y \texttt{test}. Entrenaremos con datos hasta
\texttt{2019-09-01} y los restantes para \texttt{test}:

\begin{Shaded}
\begin{Highlighting}[]
\NormalTok{train <-}\StringTok{ }\NormalTok{calidad_aire[calidad_aire}\OperatorTok{$}\NormalTok{fecha }\OperatorTok{<}\StringTok{ }\KeywordTok{as.Date}\NormalTok{(}\StringTok{"2019-09-01"}\NormalTok{),]}
\NormalTok{test <-}\StringTok{ }\NormalTok{calidad_aire[calidad_aire}\OperatorTok{$}\NormalTok{fecha }\OperatorTok{>=}\StringTok{ }\KeywordTok{as.Date}\NormalTok{(}\StringTok{"2019-09-01"}\NormalTok{),]}
\end{Highlighting}
\end{Shaded}

\hypertarget{exportaciuxf3n-de-la-informaciuxf3n}{%
\section{Exportación de la
información}\label{exportaciuxf3n-de-la-informaciuxf3n}}

Igual que hicimos en la fase anterior, alamacenamos estos datos en el
disco duro.

\begin{Shaded}
\begin{Highlighting}[]
\KeywordTok{saveRDS}\NormalTok{(train, }\DataTypeTok{file =} \StringTok{"taller_calidad_aire/data/train.RDS"}\NormalTok{)}
\KeywordTok{saveRDS}\NormalTok{(test, }\DataTypeTok{file =} \StringTok{"taller_calidad_aire/data/test.RDS"}\NormalTok{)}
\end{Highlighting}
\end{Shaded}

\end{document}
