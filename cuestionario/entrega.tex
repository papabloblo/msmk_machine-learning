\documentclass[]{article}
\usepackage{lmodern}
\usepackage{amssymb,amsmath}
\usepackage{ifxetex,ifluatex}
\usepackage{fixltx2e} % provides \textsubscript
\ifnum 0\ifxetex 1\fi\ifluatex 1\fi=0 % if pdftex
  \usepackage[T1]{fontenc}
  \usepackage[utf8]{inputenc}
\else % if luatex or xelatex
  \ifxetex
    \usepackage{mathspec}
  \else
    \usepackage{fontspec}
  \fi
  \defaultfontfeatures{Ligatures=TeX,Scale=MatchLowercase}
\fi
% use upquote if available, for straight quotes in verbatim environments
\IfFileExists{upquote.sty}{\usepackage{upquote}}{}
% use microtype if available
\IfFileExists{microtype.sty}{%
\usepackage{microtype}
\UseMicrotypeSet[protrusion]{basicmath} % disable protrusion for tt fonts
}{}
\usepackage[margin=1in]{geometry}
\usepackage{hyperref}
\hypersetup{unicode=true,
            pdftitle={Taller de ML: práctica},
            pdfauthor={Pablo Hidalgo},
            pdfborder={0 0 0},
            breaklinks=true}
\urlstyle{same}  % don't use monospace font for urls
\usepackage{color}
\usepackage{fancyvrb}
\newcommand{\VerbBar}{|}
\newcommand{\VERB}{\Verb[commandchars=\\\{\}]}
\DefineVerbatimEnvironment{Highlighting}{Verbatim}{commandchars=\\\{\}}
% Add ',fontsize=\small' for more characters per line
\usepackage{framed}
\definecolor{shadecolor}{RGB}{248,248,248}
\newenvironment{Shaded}{\begin{snugshade}}{\end{snugshade}}
\newcommand{\AlertTok}[1]{\textcolor[rgb]{0.94,0.16,0.16}{#1}}
\newcommand{\AnnotationTok}[1]{\textcolor[rgb]{0.56,0.35,0.01}{\textbf{\textit{#1}}}}
\newcommand{\AttributeTok}[1]{\textcolor[rgb]{0.77,0.63,0.00}{#1}}
\newcommand{\BaseNTok}[1]{\textcolor[rgb]{0.00,0.00,0.81}{#1}}
\newcommand{\BuiltInTok}[1]{#1}
\newcommand{\CharTok}[1]{\textcolor[rgb]{0.31,0.60,0.02}{#1}}
\newcommand{\CommentTok}[1]{\textcolor[rgb]{0.56,0.35,0.01}{\textit{#1}}}
\newcommand{\CommentVarTok}[1]{\textcolor[rgb]{0.56,0.35,0.01}{\textbf{\textit{#1}}}}
\newcommand{\ConstantTok}[1]{\textcolor[rgb]{0.00,0.00,0.00}{#1}}
\newcommand{\ControlFlowTok}[1]{\textcolor[rgb]{0.13,0.29,0.53}{\textbf{#1}}}
\newcommand{\DataTypeTok}[1]{\textcolor[rgb]{0.13,0.29,0.53}{#1}}
\newcommand{\DecValTok}[1]{\textcolor[rgb]{0.00,0.00,0.81}{#1}}
\newcommand{\DocumentationTok}[1]{\textcolor[rgb]{0.56,0.35,0.01}{\textbf{\textit{#1}}}}
\newcommand{\ErrorTok}[1]{\textcolor[rgb]{0.64,0.00,0.00}{\textbf{#1}}}
\newcommand{\ExtensionTok}[1]{#1}
\newcommand{\FloatTok}[1]{\textcolor[rgb]{0.00,0.00,0.81}{#1}}
\newcommand{\FunctionTok}[1]{\textcolor[rgb]{0.00,0.00,0.00}{#1}}
\newcommand{\ImportTok}[1]{#1}
\newcommand{\InformationTok}[1]{\textcolor[rgb]{0.56,0.35,0.01}{\textbf{\textit{#1}}}}
\newcommand{\KeywordTok}[1]{\textcolor[rgb]{0.13,0.29,0.53}{\textbf{#1}}}
\newcommand{\NormalTok}[1]{#1}
\newcommand{\OperatorTok}[1]{\textcolor[rgb]{0.81,0.36,0.00}{\textbf{#1}}}
\newcommand{\OtherTok}[1]{\textcolor[rgb]{0.56,0.35,0.01}{#1}}
\newcommand{\PreprocessorTok}[1]{\textcolor[rgb]{0.56,0.35,0.01}{\textit{#1}}}
\newcommand{\RegionMarkerTok}[1]{#1}
\newcommand{\SpecialCharTok}[1]{\textcolor[rgb]{0.00,0.00,0.00}{#1}}
\newcommand{\SpecialStringTok}[1]{\textcolor[rgb]{0.31,0.60,0.02}{#1}}
\newcommand{\StringTok}[1]{\textcolor[rgb]{0.31,0.60,0.02}{#1}}
\newcommand{\VariableTok}[1]{\textcolor[rgb]{0.00,0.00,0.00}{#1}}
\newcommand{\VerbatimStringTok}[1]{\textcolor[rgb]{0.31,0.60,0.02}{#1}}
\newcommand{\WarningTok}[1]{\textcolor[rgb]{0.56,0.35,0.01}{\textbf{\textit{#1}}}}
\usepackage{graphicx,grffile}
\makeatletter
\def\maxwidth{\ifdim\Gin@nat@width>\linewidth\linewidth\else\Gin@nat@width\fi}
\def\maxheight{\ifdim\Gin@nat@height>\textheight\textheight\else\Gin@nat@height\fi}
\makeatother
% Scale images if necessary, so that they will not overflow the page
% margins by default, and it is still possible to overwrite the defaults
% using explicit options in \includegraphics[width, height, ...]{}
\setkeys{Gin}{width=\maxwidth,height=\maxheight,keepaspectratio}
\IfFileExists{parskip.sty}{%
\usepackage{parskip}
}{% else
\setlength{\parindent}{0pt}
\setlength{\parskip}{6pt plus 2pt minus 1pt}
}
\setlength{\emergencystretch}{3em}  % prevent overfull lines
\providecommand{\tightlist}{%
  \setlength{\itemsep}{0pt}\setlength{\parskip}{0pt}}
\setcounter{secnumdepth}{0}
% Redefines (sub)paragraphs to behave more like sections
\ifx\paragraph\undefined\else
\let\oldparagraph\paragraph
\renewcommand{\paragraph}[1]{\oldparagraph{#1}\mbox{}}
\fi
\ifx\subparagraph\undefined\else
\let\oldsubparagraph\subparagraph
\renewcommand{\subparagraph}[1]{\oldsubparagraph{#1}\mbox{}}
\fi

%%% Use protect on footnotes to avoid problems with footnotes in titles
\let\rmarkdownfootnote\footnote%
\def\footnote{\protect\rmarkdownfootnote}

%%% Change title format to be more compact
\usepackage{titling}

% Create subtitle command for use in maketitle
\providecommand{\subtitle}[1]{
  \posttitle{
    \begin{center}\large#1\end{center}
    }
}

\setlength{\droptitle}{-2em}

  \title{Taller de ML: práctica}
    \pretitle{\vspace{\droptitle}\centering\huge}
  \posttitle{\par}
    \author{Pablo Hidalgo}
    \preauthor{\centering\large\emph}
  \postauthor{\par}
      \predate{\centering\large\emph}
  \postdate{\par}
    \date{Entrega: 21/10/2019}


\begin{document}
\maketitle

Esta práctica consiste en la aplicación de los contenidos que se han
visto en el taller de ML. \textbf{La fecha límite de entrega es el 21 de
octubre.} La entrega deberá consistir en

\begin{itemize}
\tightlist
\item
  \textbf{script de R} donde aparezca todo el código empleado para
  realizar la práctica. Es recomendable utilizar comentarios para
  aclarar el código (recuerda, para escribir un comentario empieza por
  la línea \texttt{\#}),
\item
  \textbf{documento en pdf} donde aparezcan los gráficos, resultados y
  se hagan los comentarios apropiados. Se puede hacer en cualquier
  editor de texto (por ejemplo, word) y luego exportase a pdf.
\end{itemize}

\hypertarget{los-datos}{%
\section{Los datos}\label{los-datos}}

Para esta entrega vamos a utilizar unos datos muy similares a los vistos
en clase. Son datos de \textbf{ventas de casas} pero de otra zona de
Estados Unidos, en concreto del Condado de King (Washington).

Las variables de este conjunto de datos son:

\begin{itemize}
\tightlist
\item
  \texttt{id}: identificador único de cada casa vendida.
\item
  \texttt{date}: fecha de venta.
\item
  \texttt{price}: precio de venta (en dólares).
\item
  \texttt{bedrooms}: número de habitaciones.
\item
  \texttt{bathrooms}: número de baños donde \(0.5\) significa una
  habitación con aseo (baño sin ducha).
\item
  \texttt{sqft\_living}: área habitable de la casa (en pies cuadrados).
\item
  \texttt{sqft\_lot}: área de la parcela (en pies cuadrados).
\item
  \texttt{floors}: número de plantas de la vivienda.
\item
  \texttt{waterfront}: variable binaria (\emph{dummy}) indicando si la
  vivienda tiene vistas al mar o no.
\item
  \texttt{view}: índice (0-4) de las vistas.
\item
  \texttt{condition}: índice (1-5) del estado de la vivienda.
\item
  \texttt{grade}: índice (1-13) de la calidad de la construcción y
  diseño.
\item
  \texttt{sqft\_above}: área de la parte que está por encima del nivel
  del suelo (en pies cuadrados).
\item
  \texttt{sqft\_basement}: área del sótano (en pies cuadrados).
\item
  \texttt{yr\_built}: año de construcción.
\item
  \texttt{yr\_renovated}: año de la última reforma.
\item
  \texttt{zipcode}: código postal.
\item
  \texttt{lat}: latitud.
\item
  \texttt{long}: longitud.
\item
  \texttt{sqft\_living15}: área de vivienda media de las 15 casas más
  próximas (en pies cuadrados).
\item
  \texttt{sqft\_lot15}: área de parcela media de las 15 casas más
  próximas (en pies cuadrados).
\end{itemize}

\hypertarget{desarrollo-de-la-practica.}{%
\section{Desarrollo de la práctica.}\label{desarrollo-de-la-practica.}}

Comienza cargando el archivo \texttt{.csv} en la memoria de R. Para
ello, recuerda que puedes utilizar la función \texttt{read\_csv()}
(previamente, deberías haber cargado la librería \texttt{tidyverse}).

Inspecciona las variables del conjunto de datos mediante la función
\texttt{skim()} del paquete \texttt{skimr}. Comenta el resultado
destacando aquello que encuentres más relevante.

Al tratarse, de nuevo, de un conjunto de datos de Estados Unidos, las
variables de superficie vienen expresadas en pies cuadrados.
\textbf{Convierte todas las variables de superficie a metros cuadrados.}

Estos datos contienen la ubicación (latitud y longitud) de cada venta.
Existen paquetes específicos para poder visualizar esta información.
\textbf{Instala el paquete \texttt{leaflet} y ejecuta la siguiente
sentencia:}

\begin{Shaded}
\begin{Highlighting}[]
\KeywordTok{library}\NormalTok{(leaflet)}
\CommentTok{# Susituye nombre_datos por el nombre que tengan tus datos }
\KeywordTok{leaflet}\NormalTok{(nombre_datos) }\OperatorTok\StringTok{ }\KeywordTok{addTiles}\NormalTok{() }\OperatorTok\StringTok{ }\KeywordTok{addCircleMarkers}\NormalTok{()}
\end{Highlighting}
\end{Shaded}

\textbf{Representa la relación que hay entre la superificie de cada
vivienda y su precio con un gráfico de puntos. Cambia el título y el
nombre de los ejes para hacerlo más fácil de leer.}

\textbf{Divide el conjunto de datos en \texttt{train} y \texttt{test} de
forma que sean el 80\% y el 20\% de los datos, respectivamente.}

\textbf{Justifica qué variables del conjunto de datos debería excluirse
de un modelo para predecir el precio de venta.}

Ajusta primero un modelo de regresión lineal utilizando solamente la
superficie de la casa para predecir el precio de venta (llama a este
modelo \texttt{lm1}). Entrena otro modelo utilizando la superficie y la
variable \texttt{grade} (llámalo \texttt{lm2}). \textbf{Comenta ambos
modelos y sus diferencias.}

Ejecuta las siguientes líneas de código:

\begin{Shaded}
\begin{Highlighting}[]
\CommentTok{#Cambia datos_test por el nombre que le hayas dado a test}
\NormalTok{pred_test <-}\StringTok{ }\KeywordTok{select}\NormalTok{(datos_test, price)}
\NormalTok{pred_test}\OperatorTok{$}\NormalTok{pred_lm1 <-}\StringTok{ }\KeywordTok{predict}\NormalTok{(lm1, }\DataTypeTok{newdata =}\NormalTok{ datos_test)}
\NormalTok{pred_test}\OperatorTok{$}\NormalTok{pred_lm2 <-}\StringTok{ }\KeywordTok{predict}\NormalTok{(lm2, }\DataTypeTok{newdata =}\NormalTok{ datos_test)}
\NormalTok{rmse <-}\StringTok{ }\ControlFlowTok{function}\NormalTok{(price, pred) }\KeywordTok{sqrt}\NormalTok{(}\KeywordTok{mean}\NormalTok{((price }\OperatorTok{-}\StringTok{ }\NormalTok{pred)}\OperatorTok{^}\DecValTok{2}\NormalTok{))}
\NormalTok{error <-}\StringTok{ }\KeywordTok{tibble}\NormalTok{(}
  \DataTypeTok{modelo =} \KeywordTok{c}\NormalTok{(}\StringTok{"lm1"}\NormalTok{, }\StringTok{"lm2"}\NormalTok{),}
  \DataTypeTok{error =} \KeywordTok{c}\NormalTok{(}\KeywordTok{rmse}\NormalTok{(pred_test}\OperatorTok{$}\NormalTok{price, pred_test}\OperatorTok{$}\NormalTok{pred_lm1), }\KeywordTok{rmse}\NormalTok{(pred_test}\OperatorTok{$}\NormalTok{price, pred_test}\OperatorTok{$}\NormalTok{pred_lm2))}
\NormalTok{)}
\end{Highlighting}
\end{Shaded}

Representa el gráfico de residuos de ambos modelos y coméntalos.

En clase hemos visto el modelo de Gradient Boosting, uno de los modelos
de \emph{machine learning} más utilizados. Ejecuta el siguiente código
cambiando lo necesario:

\begin{Shaded}
\begin{Highlighting}[]
\NormalTok{train_x <-}\StringTok{ }\KeywordTok{select}\NormalTok{(datos_train, }
                  \OperatorTok{-}\NormalTok{nombre_variable_precio, }
                  \OperatorTok{-}\NormalTok{nombre_variable_id,}
                  \OperatorTok{-}\NormalTok{nombre_variable_fecha) }\OperatorTok\StringTok{ }
\StringTok{  }\KeywordTok{as.matrix}\NormalTok{()}

\NormalTok{test_x <-}\StringTok{ }\KeywordTok{select}\NormalTok{(datos_test, }
                 \OperatorTok{-}\NormalTok{nombre_variable_precio, }
                 \OperatorTok{-}\NormalTok{nombre_variable_id, }
                 \OperatorTok{-}\NormalTok{nombre_variable_fecha) }\OperatorTok\StringTok{ }
\StringTok{  }\KeywordTok{as.matrix}\NormalTok{()}
\NormalTok{test_x[}\KeywordTok{is.na}\NormalTok{(test_x)] <-}\StringTok{ }\DecValTok{0}


\KeywordTok{library}\NormalTok{(xgboost)}

\NormalTok{gradient_boost <-}\StringTok{ }\KeywordTok{xgboost}\NormalTok{(}\DataTypeTok{data =}\NormalTok{ train_x, }
                          \DataTypeTok{label =}\NormalTok{ datos_train}\OperatorTok{$}\NormalTok{nombre_variable_precio,}
                          \DataTypeTok{nrounds =} \DecValTok{1000}\NormalTok{,}
                          \DataTypeTok{params =} \KeywordTok{list}\NormalTok{(}\DataTypeTok{eta =} \FloatTok{0.01}\NormalTok{, }
                                        \DataTypeTok{max_depth =} \DecValTok{3}\NormalTok{,}
                                        \DataTypeTok{colsample_bytree =} \FloatTok{0.75}
\NormalTok{                                        )}
\NormalTok{                          )}




\NormalTok{pred_test}\OperatorTok{$}\NormalTok{pred_bst <-}\StringTok{ }\KeywordTok{predict}\NormalTok{(gradient_boost, test_x)}

\NormalTok{error <-}\StringTok{ }\NormalTok{error }\OperatorTok\StringTok{ }
\StringTok{  }\KeywordTok{bind_rows}\NormalTok{(}\KeywordTok{c}\NormalTok{(}\DataTypeTok{modelo =} \StringTok{"bst"}\NormalTok{, }\DataTypeTok{error =} \KeywordTok{rmse}\NormalTok{(pred_test}\OperatorTok{$}\NormalTok{price, pred_test}\OperatorTok{$}\NormalTok{pred_bst)))}
\end{Highlighting}
\end{Shaded}

De los modelos que hemos entrenado, \textbf{¿con cuál te quedarías y por
qué?}

\hypertarget{explicatividad}{%
\section{Explicatividad}\label{explicatividad}}

Esribe y comenta los gráficos que consideres necesarios para interpretar
el algoritmo de \emph{Gradient boosting}.


\end{document}
